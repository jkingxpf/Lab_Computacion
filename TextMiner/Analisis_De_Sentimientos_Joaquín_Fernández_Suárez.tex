% Options for packages loaded elsewhere
\PassOptionsToPackage{unicode}{hyperref}
\PassOptionsToPackage{hyphens}{url}
%
\documentclass[
]{article}
\usepackage{amsmath,amssymb}
\usepackage{lmodern}
\usepackage{iftex}
\ifPDFTeX
  \usepackage[T1]{fontenc}
  \usepackage[utf8]{inputenc}
  \usepackage{textcomp} % provide euro and other symbols
\else % if luatex or xetex
  \usepackage{unicode-math}
  \defaultfontfeatures{Scale=MatchLowercase}
  \defaultfontfeatures[\rmfamily]{Ligatures=TeX,Scale=1}
\fi
% Use upquote if available, for straight quotes in verbatim environments
\IfFileExists{upquote.sty}{\usepackage{upquote}}{}
\IfFileExists{microtype.sty}{% use microtype if available
  \usepackage[]{microtype}
  \UseMicrotypeSet[protrusion]{basicmath} % disable protrusion for tt fonts
}{}
\makeatletter
\@ifundefined{KOMAClassName}{% if non-KOMA class
  \IfFileExists{parskip.sty}{%
    \usepackage{parskip}
  }{% else
    \setlength{\parindent}{0pt}
    \setlength{\parskip}{6pt plus 2pt minus 1pt}}
}{% if KOMA class
  \KOMAoptions{parskip=half}}
\makeatother
\usepackage{xcolor}
\usepackage[margin=1in]{geometry}
\usepackage{color}
\usepackage{fancyvrb}
\newcommand{\VerbBar}{|}
\newcommand{\VERB}{\Verb[commandchars=\\\{\}]}
\DefineVerbatimEnvironment{Highlighting}{Verbatim}{commandchars=\\\{\}}
% Add ',fontsize=\small' for more characters per line
\usepackage{framed}
\definecolor{shadecolor}{RGB}{248,248,248}
\newenvironment{Shaded}{\begin{snugshade}}{\end{snugshade}}
\newcommand{\AlertTok}[1]{\textcolor[rgb]{0.94,0.16,0.16}{#1}}
\newcommand{\AnnotationTok}[1]{\textcolor[rgb]{0.56,0.35,0.01}{\textbf{\textit{#1}}}}
\newcommand{\AttributeTok}[1]{\textcolor[rgb]{0.77,0.63,0.00}{#1}}
\newcommand{\BaseNTok}[1]{\textcolor[rgb]{0.00,0.00,0.81}{#1}}
\newcommand{\BuiltInTok}[1]{#1}
\newcommand{\CharTok}[1]{\textcolor[rgb]{0.31,0.60,0.02}{#1}}
\newcommand{\CommentTok}[1]{\textcolor[rgb]{0.56,0.35,0.01}{\textit{#1}}}
\newcommand{\CommentVarTok}[1]{\textcolor[rgb]{0.56,0.35,0.01}{\textbf{\textit{#1}}}}
\newcommand{\ConstantTok}[1]{\textcolor[rgb]{0.00,0.00,0.00}{#1}}
\newcommand{\ControlFlowTok}[1]{\textcolor[rgb]{0.13,0.29,0.53}{\textbf{#1}}}
\newcommand{\DataTypeTok}[1]{\textcolor[rgb]{0.13,0.29,0.53}{#1}}
\newcommand{\DecValTok}[1]{\textcolor[rgb]{0.00,0.00,0.81}{#1}}
\newcommand{\DocumentationTok}[1]{\textcolor[rgb]{0.56,0.35,0.01}{\textbf{\textit{#1}}}}
\newcommand{\ErrorTok}[1]{\textcolor[rgb]{0.64,0.00,0.00}{\textbf{#1}}}
\newcommand{\ExtensionTok}[1]{#1}
\newcommand{\FloatTok}[1]{\textcolor[rgb]{0.00,0.00,0.81}{#1}}
\newcommand{\FunctionTok}[1]{\textcolor[rgb]{0.00,0.00,0.00}{#1}}
\newcommand{\ImportTok}[1]{#1}
\newcommand{\InformationTok}[1]{\textcolor[rgb]{0.56,0.35,0.01}{\textbf{\textit{#1}}}}
\newcommand{\KeywordTok}[1]{\textcolor[rgb]{0.13,0.29,0.53}{\textbf{#1}}}
\newcommand{\NormalTok}[1]{#1}
\newcommand{\OperatorTok}[1]{\textcolor[rgb]{0.81,0.36,0.00}{\textbf{#1}}}
\newcommand{\OtherTok}[1]{\textcolor[rgb]{0.56,0.35,0.01}{#1}}
\newcommand{\PreprocessorTok}[1]{\textcolor[rgb]{0.56,0.35,0.01}{\textit{#1}}}
\newcommand{\RegionMarkerTok}[1]{#1}
\newcommand{\SpecialCharTok}[1]{\textcolor[rgb]{0.00,0.00,0.00}{#1}}
\newcommand{\SpecialStringTok}[1]{\textcolor[rgb]{0.31,0.60,0.02}{#1}}
\newcommand{\StringTok}[1]{\textcolor[rgb]{0.31,0.60,0.02}{#1}}
\newcommand{\VariableTok}[1]{\textcolor[rgb]{0.00,0.00,0.00}{#1}}
\newcommand{\VerbatimStringTok}[1]{\textcolor[rgb]{0.31,0.60,0.02}{#1}}
\newcommand{\WarningTok}[1]{\textcolor[rgb]{0.56,0.35,0.01}{\textbf{\textit{#1}}}}
\usepackage{graphicx}
\makeatletter
\def\maxwidth{\ifdim\Gin@nat@width>\linewidth\linewidth\else\Gin@nat@width\fi}
\def\maxheight{\ifdim\Gin@nat@height>\textheight\textheight\else\Gin@nat@height\fi}
\makeatother
% Scale images if necessary, so that they will not overflow the page
% margins by default, and it is still possible to overwrite the defaults
% using explicit options in \includegraphics[width, height, ...]{}
\setkeys{Gin}{width=\maxwidth,height=\maxheight,keepaspectratio}
% Set default figure placement to htbp
\makeatletter
\def\fps@figure{htbp}
\makeatother
\setlength{\emergencystretch}{3em} % prevent overfull lines
\providecommand{\tightlist}{%
  \setlength{\itemsep}{0pt}\setlength{\parskip}{0pt}}
\setcounter{secnumdepth}{-\maxdimen} % remove section numbering
\ifLuaTeX
  \usepackage{selnolig}  % disable illegal ligatures
\fi
\IfFileExists{bookmark.sty}{\usepackage{bookmark}}{\usepackage{hyperref}}
\IfFileExists{xurl.sty}{\usepackage{xurl}}{} % add URL line breaks if available
\urlstyle{same} % disable monospaced font for URLs
\hypersetup{
  pdftitle={Proyecto de Text Mining - Análisis de comentarios.},
  pdfauthor={Joaquín Fernández Suárez},
  hidelinks,
  pdfcreator={LaTeX via pandoc}}

\title{Proyecto de Text Mining - Análisis de comentarios.}
\author{Joaquín Fernández Suárez}
\date{}

\begin{document}
\maketitle

\hypertarget{introducciuxf3n.}{%
\section{Introducción.}\label{introducciuxf3n.}}

En el siguiente proyecto se realizara un análisis sobre los comentarios
de la siguiente
\href{https://www.instagram.com/p/CgemyPPKAGA/?utm_source=ig_web_copy_link&igshid=MzRlODBiNWFlZA==}(publicación)
perteneciente al influencer ``elXokas'' con el objetivo de descubrir la
opinión de sus seguidores al respecto de la publicación.

\hypertarget{quien-es-elxokas}{%
\subsubsection{¿Quien es elXokas?}\label{quien-es-elxokas}}

El influenser y streamer elXokas es un creador de contenido gallego
famoso por sus comentarios directos y controvertidos, que le han
provocado estar involucrado en diferentes discusiones y polemicas en el
último año aunque su contenido principal no sea relacionado a la
política o la prensa amarilla.

Por estas cualidades vamos a elegir a este usuario de Instagram ya que
se pueden sacar información interesante de la opinión de sus seguidores
respecto a una publicación totalmente aleatoria.

\hypertarget{publicaciuxf3n-escogida.}{%
\subsubsection{Publicación escogida.}\label{publicaciuxf3n-escogida.}}

La publicación antes señalada muestra a elXokas junto a uno de los
mayores influencers de habla hispana elRubius, un personaje muy querido
por la comunidad. Debido a el caracter amistoso de esta publicación en
contraposición a la fama de este influencer parece interesante ver que
es lo que desean señalar sus seguidores al respecto.

\hypertarget{librerias-usadas.}{%
\subsubsection{Librerias usadas.}\label{librerias-usadas.}}

Para este proyecto usaremos la libreria \texttt{tidyverse} y
\texttt{tidytext} para analizar el texto optenido. Tambien se usara la
librería \texttt{lubridate} para trabajar con las fechas

\begin{Shaded}
\begin{Highlighting}[]
\FunctionTok{library}\NormalTok{(tidyverse)}
\end{Highlighting}
\end{Shaded}

\begin{verbatim}
## Warning: package 'ggplot2' was built under R version 4.2.3
\end{verbatim}

\begin{verbatim}
## Warning: package 'tibble' was built under R version 4.2.3
\end{verbatim}

\begin{verbatim}
## Warning: package 'dplyr' was built under R version 4.2.3
\end{verbatim}

\begin{verbatim}
## -- Attaching core tidyverse packages ------------------------ tidyverse 2.0.0 --
## v dplyr     1.1.1     v readr     2.1.4
## v forcats   1.0.0     v stringr   1.5.0
## v ggplot2   3.4.2     v tibble    3.2.1
## v lubridate 1.9.2     v tidyr     1.3.0
## v purrr     1.0.1     
## -- Conflicts ------------------------------------------ tidyverse_conflicts() --
## x dplyr::filter() masks stats::filter()
## x dplyr::lag()    masks stats::lag()
## i Use the ]8;;http://conflicted.r-lib.org/conflicted package]8;; to force all conflicts to become errors
\end{verbatim}

\begin{Shaded}
\begin{Highlighting}[]
\FunctionTok{library}\NormalTok{(tidytext)}
\end{Highlighting}
\end{Shaded}

\begin{verbatim}
## Warning: package 'tidytext' was built under R version 4.2.3
\end{verbatim}

\begin{Shaded}
\begin{Highlighting}[]
\FunctionTok{library}\NormalTok{(lubridate)}
\end{Highlighting}
\end{Shaded}

\hypertarget{obtenciuxf3n-de-los-comentarios.}{%
\subsubsection{Obtención de los
comentarios.}\label{obtenciuxf3n-de-los-comentarios.}}

Para optener los comentarios de la aplicación Instagram se ha usado la
aplicación \textbf{IGCommentExport} que dada la url de la publicación
nos proporciona un fichero .csv con los comentarios de la publicación.
Debido de que tiene una versión premium solo nos permite obtener como
máximo cien comentarios.

Estos serian los comentarios sin realizar ningun tratamiento.

\begin{Shaded}
\begin{Highlighting}[]
\NormalTok{comentarios\_ig }\OtherTok{\textless{}{-}} \FunctionTok{read\_csv}\NormalTok{(}\StringTok{"\textasciitilde{}/R/Lab\_Computacion/TextMiner/IGCommentExport\_CgemyPPKAGA\_100\_comments.csv"}\NormalTok{)}
\end{Highlighting}
\end{Shaded}

\begin{verbatim}
## Rows: 100 Columns: 7
## -- Column specification --------------------------------------------------------
## Delimiter: ","
## chr (7): ="User Id", ="Username", ="Comment Id", ="Comment Text", ="Profile ...
## 
## i Use `spec()` to retrieve the full column specification for this data.
## i Specify the column types or set `show_col_types = FALSE` to quiet this message.
\end{verbatim}

\begin{Shaded}
\begin{Highlighting}[]
\FunctionTok{head}\NormalTok{(comentarios\_ig,}\DecValTok{10}\NormalTok{)}
\end{Highlighting}
\end{Shaded}

\begin{verbatim}
## # A tibble: 10 x 7
##    `="User Id"` `="Username"` `="Comment Id"` `="Comment Text"` `="Profile URL"`
##    <chr>        <chr>         <chr>           <chr>             <chr>           
##  1 "=\"2606320~ "=\"jonathan~ "=\"1797364506~ "=\"\U0001f91d\U~ "=\"https://www~
##  2 "=\"2606320~ "=\"jonathan~ "=\"1805994243~ "=\"Donde estas\~ "=\"https://www~
##  3 "=\"2606320~ "=\"jonathan~ "=\"1824262256~ "=\"6/7 donde es~ "=\"https://www~
##  4 "=\"2606320~ "=\"jonathan~ "=\"1795451530~ "=\"Ya 6/7 a la ~ "=\"https://www~
##  5 "=\"2606320~ "=\"jonathan~ "=\"1798258203~ "=\"6/7 genialis~ "=\"https://www~
##  6 "=\"2606320~ "=\"jonathan~ "=\"1797926197~ "=\"@prokdts chu~ "=\"https://www~
##  7 "=\"8973589~ "=\"fr.1967s~ "=\"1795832291~ "=\"@le_saint_av~ "=\"https://www~
##  8 "=\"3168273~ "=\"kian_far~ "=\"1819087034~ "=\"@fr.1967s ru~ "=\"https://www~
##  9 "=\"4800785~ "=\"mendo4fu~ "=\"1796345619~ "=\"@anthony.dav~ "=\"https://www~
## 10 "=\"1519807~ "=\"jtavaler~ "=\"1793164868~ "=\"Xokas ya te ~ "=\"https://www~
## # i 2 more variables: `="Profile Pic URL"` <chr>, `="Date"` <chr>
\end{verbatim}

\hypertarget{tratamiento-de-los-datos.}{%
\section{Tratamiento de los datos.}\label{tratamiento-de-los-datos.}}

Una vez descargados los comentarios, se tratarán para obtener
información con la que poder trabajar.

Se comienza eliminando las columnas del dataset que no aportan
infromación útil para la investigación.

\begin{Shaded}
\begin{Highlighting}[]
\FunctionTok{colnames}\NormalTok{(comentarios\_ig)}
\end{Highlighting}
\end{Shaded}

\begin{verbatim}
## [1] "=\"User Id\""         "=\"Username\""        "=\"Comment Id\""     
## [4] "=\"Comment Text\""    "=\"Profile URL\""     "=\"Profile Pic URL\""
## [7] "=\"Date\""
\end{verbatim}

Guardaremos los siguientes datos, que conforman el nombre del perfil que
realizo el comentario y el texto del mismo.

\begin{Shaded}
\begin{Highlighting}[]
\NormalTok{comentarios }\OtherTok{\textless{}{-}}\NormalTok{ comentarios\_ig }\SpecialCharTok{\%\textgreater{}\%} \FunctionTok{select}\NormalTok{(}\FunctionTok{c}\NormalTok{(}\StringTok{"=}\SpecialCharTok{\textbackslash{}"}\StringTok{Username}\SpecialCharTok{\textbackslash{}"}\StringTok{"}\NormalTok{, }\StringTok{"=}\SpecialCharTok{\textbackslash{}"}\StringTok{Comment Text}\SpecialCharTok{\textbackslash{}"}\StringTok{"}\NormalTok{, }\StringTok{"=}\SpecialCharTok{\textbackslash{}"}\StringTok{Date}\SpecialCharTok{\textbackslash{}"}\StringTok{"}\NormalTok{))}
\end{Highlighting}
\end{Shaded}

Y tratamos el dataset para poder trabajar eficientemente con él.

\begin{Shaded}
\begin{Highlighting}[]
\DocumentationTok{\#\#Renombramos las variables}
\FunctionTok{colnames}\NormalTok{(comentarios)}
\end{Highlighting}
\end{Shaded}

\begin{verbatim}
## [1] "=\"Username\""     "=\"Comment Text\"" "=\"Date\""
\end{verbatim}

\begin{Shaded}
\begin{Highlighting}[]
\NormalTok{comentarios }\OtherTok{\textless{}{-}}\NormalTok{ comentarios }\SpecialCharTok{\%\textgreater{}\%} \FunctionTok{rename}\NormalTok{(}\StringTok{"Username"} \OtherTok{=} \StringTok{"=}\SpecialCharTok{\textbackslash{}"}\StringTok{Username}\SpecialCharTok{\textbackslash{}"}\StringTok{"}\NormalTok{,}
                                      \StringTok{"Text"} \OtherTok{=} \StringTok{"=}\SpecialCharTok{\textbackslash{}"}\StringTok{Comment Text}\SpecialCharTok{\textbackslash{}"}\StringTok{"}\NormalTok{ ,}
                                      \StringTok{"Date"} \OtherTok{=} \StringTok{"=}\SpecialCharTok{\textbackslash{}"}\StringTok{Date}\SpecialCharTok{\textbackslash{}"}\StringTok{"}\NormalTok{)}

\DocumentationTok{\#\#La variable "Date" la transformaremos a clase date.}

\DocumentationTok{\#\#Usamos gsub para eliminar los caracteres que no deseamos.}

\NormalTok{comentarios}\SpecialCharTok{$}\NormalTok{Date }\OtherTok{\textless{}{-}} \FunctionTok{gsub}\NormalTok{(}\StringTok{"[\^{}0{-}9a{-}zA{-}Z/: ]"}\NormalTok{, }\StringTok{""}\NormalTok{, comentarios}\SpecialCharTok{$}\NormalTok{Date)}
  
\NormalTok{comentarios}\SpecialCharTok{$}\NormalTok{Date }\OtherTok{\textless{}{-}} \FunctionTok{as.Date}\NormalTok{(comentarios}\SpecialCharTok{$}\NormalTok{Date, }\AttributeTok{tryFormats =} \FunctionTok{c}\NormalTok{(}\StringTok{"\%d{-}\%m{-}\%Y"}\NormalTok{,}\StringTok{"\%d/\%m/\%Y"}\NormalTok{))}

\DocumentationTok{\#\#Eliminamos caracteres = y " no necesarios de la variable Username.}

\NormalTok{comentarios}\SpecialCharTok{$}\NormalTok{Username }\OtherTok{\textless{}{-}} \FunctionTok{gsub}\NormalTok{(}\StringTok{"[=}\SpecialCharTok{\textbackslash{}\textbackslash{}\textbackslash{}"}\StringTok{]"}\NormalTok{, }\StringTok{""}\NormalTok{, comentarios}\SpecialCharTok{$}\NormalTok{Username)}
\end{Highlighting}
\end{Shaded}

A continuación crearemos un nuevo dataset con tokens, conformados con
las palabras que estan contenidas en la variable \texttt{Text} del nuevo
dataset.

\begin{Shaded}
\begin{Highlighting}[]
\NormalTok{token\_text }\OtherTok{\textless{}{-}}\NormalTok{ comentarios }\SpecialCharTok{\%\textgreater{}\%} 
  \FunctionTok{unnest\_tokens}\NormalTok{(word, Text)}

\FunctionTok{dim}\NormalTok{(token\_text)}
\end{Highlighting}
\end{Shaded}

\begin{verbatim}
## [1] 1159    3
\end{verbatim}

A continuación se eliminarán las palabras vacías, palabras que no nos
aportan información relevante.

Es necesario destacar que \texttt{tidytext} no tiene lista de palabras
vacías en español , pot tanto, usaremos las palabras vacias contenidas
en el paquete \texttt{tm}.

\begin{Shaded}
\begin{Highlighting}[]
\FunctionTok{library}\NormalTok{(tm)}
\end{Highlighting}
\end{Shaded}

\begin{verbatim}
## Warning: package 'tm' was built under R version 4.2.3
\end{verbatim}

\begin{verbatim}
## Loading required package: NLP
\end{verbatim}

\begin{verbatim}
## 
## Attaching package: 'NLP'
\end{verbatim}

\begin{verbatim}
## The following object is masked from 'package:ggplot2':
## 
##     annotate
\end{verbatim}

\begin{Shaded}
\begin{Highlighting}[]
\NormalTok{palabras\_vacias\_español }\OtherTok{\textless{}{-}} \FunctionTok{bind\_rows}\NormalTok{(stop\_words,}
                                     \FunctionTok{data\_frame}\NormalTok{(}\AttributeTok{word =}\NormalTok{ tm}\SpecialCharTok{::}\FunctionTok{stopwords}\NormalTok{(}\StringTok{"spanish"}\NormalTok{),}\AttributeTok{lexicon =} \StringTok{"custom"}\NormalTok{))}
\end{Highlighting}
\end{Shaded}

\begin{verbatim}
## Warning: `data_frame()` was deprecated in tibble 1.1.0.
## i Please use `tibble()` instead.
## This warning is displayed once every 8 hours.
## Call `lifecycle::last_lifecycle_warnings()` to see where this warning was
## generated.
\end{verbatim}

Se eliminan las palabras vacías del dataset.

\begin{Shaded}
\begin{Highlighting}[]
\NormalTok{token\_text }\OtherTok{\textless{}{-}}\NormalTok{ token\_text }\SpecialCharTok{\%\textgreater{}\%} 
  \FunctionTok{anti\_join}\NormalTok{(palabras\_vacias\_español)}
\end{Highlighting}
\end{Shaded}

\begin{verbatim}
## Joining with `by = join_by(word)`
\end{verbatim}

\begin{Shaded}
\begin{Highlighting}[]
\FunctionTok{dim}\NormalTok{(token\_text)}
\end{Highlighting}
\end{Shaded}

\begin{verbatim}
## [1] 591   3
\end{verbatim}

Con el dataset tratado pasaremos al análisis y visualización de los
datos.

\hypertarget{visualizaciuxf3n.}{%
\section{Visualización.}\label{visualizaciuxf3n.}}

Primero se comenzará visualizando las palabras que más aparecen en los
comentarios.

\begin{Shaded}
\begin{Highlighting}[]
\NormalTok{token\_text }\SpecialCharTok{\%\textgreater{}\%} \FunctionTok{count}\NormalTok{(word, }\AttributeTok{sort =}\NormalTok{ T)}
\end{Highlighting}
\end{Shaded}

\begin{verbatim}
## # A tibble: 453 x 2
##    word               n
##    <chr>          <int>
##  1 si                11
##  2 xokas              9
##  3 _delta.rp          5
##  4 alguien            5
##  5 arribasangel10     5
##  6 así                5
##  7 mejor              5
##  8 creo               4
##  9 da                 4
## 10 hace               4
## # i 443 more rows
\end{verbatim}

Viendo la cantidad de palabras únicas que existen en los comentarios, es
más interesante hacer un análisis de sentimientos, que nos permita
analizar la cantidad de apoyo o hate recibe este influencer.

\hypertarget{anuxe1lisis-de-sentimientos.}{%
\section{Análisis de sentimientos.}\label{anuxe1lisis-de-sentimientos.}}

Primero obtendremos un diccionario de sentimientos en español para poder
trabajar con el idioma.

\begin{Shaded}
\begin{Highlighting}[]
\FunctionTok{library}\NormalTok{(here)}
\end{Highlighting}
\end{Shaded}

\begin{verbatim}
## Warning: package 'here' was built under R version 4.2.3
\end{verbatim}

\begin{verbatim}
## here() starts at C:/Users/jking/Documents/R/Lab_Computacion
\end{verbatim}

\begin{Shaded}
\begin{Highlighting}[]
\NormalTok{positive\_words }\OtherTok{\textless{}{-}}
  \FunctionTok{read\_csv}\NormalTok{(here}\SpecialCharTok{::}\FunctionTok{here}\NormalTok{(}\StringTok{"TextMiner/sentiment{-}lexicons/positive\_words\_es.txt"}\NormalTok{), }\AttributeTok{col\_names =} \StringTok{"word"}\NormalTok{) }\SpecialCharTok{\%\textgreater{}\%}
  \FunctionTok{mutate}\NormalTok{(}\AttributeTok{sentiment =} \StringTok{"positive"}\NormalTok{)}
\end{Highlighting}
\end{Shaded}

\begin{verbatim}
## Rows: 1555 Columns: 1
\end{verbatim}

\begin{verbatim}
## -- Column specification --------------------------------------------------------
## Delimiter: ","
## chr (1): word
## 
## i Use `spec()` to retrieve the full column specification for this data.
## i Specify the column types or set `show_col_types = FALSE` to quiet this message.
\end{verbatim}

\begin{Shaded}
\begin{Highlighting}[]
\NormalTok{negative\_words }\OtherTok{\textless{}{-}}
  \FunctionTok{read\_csv}\NormalTok{(here}\SpecialCharTok{::}\FunctionTok{here}\NormalTok{(}\StringTok{"TextMiner/sentiment{-}lexicons/negative\_words\_es.txt"}\NormalTok{), }\AttributeTok{col\_names =} \StringTok{"word"}\NormalTok{) }\SpecialCharTok{\%\textgreater{}\%}
  \FunctionTok{mutate}\NormalTok{(}\AttributeTok{sentiment =} \StringTok{"negative"}\NormalTok{)}
\end{Highlighting}
\end{Shaded}

\begin{verbatim}
## Rows: 2720 Columns: 1
## -- Column specification --------------------------------------------------------
## Delimiter: ","
## chr (1): word
## 
## i Use `spec()` to retrieve the full column specification for this data.
## i Specify the column types or set `show_col_types = FALSE` to quiet this message.
\end{verbatim}

\begin{Shaded}
\begin{Highlighting}[]
\NormalTok{sentiment\_words }\OtherTok{\textless{}{-}} \FunctionTok{bind\_rows}\NormalTok{(positive\_words, negative\_words)}
\end{Highlighting}
\end{Shaded}

A continuación se usará \texttt{inner\_join}para enlazar las etiquetas
del diccionario a laspalabras.

\begin{Shaded}
\begin{Highlighting}[]
\NormalTok{sentimientos\_bing }\OtherTok{\textless{}{-}}\NormalTok{ token\_text }\SpecialCharTok{\%\textgreater{}\%} \FunctionTok{inner\_join}\NormalTok{(sentiment\_words)}
\end{Highlighting}
\end{Shaded}

\begin{verbatim}
## Joining with `by = join_by(word)`
\end{verbatim}

Se mostrarán la cantidad de palabras negativas y positivas.

\begin{Shaded}
\begin{Highlighting}[]
\NormalTok{sentimientos\_bing }\SpecialCharTok{\%\textgreater{}\%}
  \FunctionTok{summarise}\NormalTok{(}\AttributeTok{Negativos =} \FunctionTok{sum}\NormalTok{(sentiment }\SpecialCharTok{==} \StringTok{"negative"}\NormalTok{), }
            \AttributeTok{Positivos=} \FunctionTok{sum}\NormalTok{(sentiment }\SpecialCharTok{==} \StringTok{"positive"}\NormalTok{))}
\end{Highlighting}
\end{Shaded}

\begin{verbatim}
## # A tibble: 1 x 2
##   Negativos Positivos
##       <int>     <int>
## 1        29        42
\end{verbatim}

\begin{Shaded}
\begin{Highlighting}[]
\NormalTok{sentimientos\_bing }\SpecialCharTok{\%\textgreater{}\%}
  \FunctionTok{group\_by}\NormalTok{(sentiment) }\SpecialCharTok{\%\textgreater{}\%}
  \FunctionTok{count}\NormalTok{(word, }\AttributeTok{sort =} \ConstantTok{TRUE}\NormalTok{) }\SpecialCharTok{\%\textgreater{}\%}
  \FunctionTok{ggplot}\NormalTok{(}\FunctionTok{aes}\NormalTok{(word, n, }\AttributeTok{fill =}\NormalTok{ sentiment)) }\SpecialCharTok{+}
  \FunctionTok{geom\_col}\NormalTok{(}\AttributeTok{show.legend =} \ConstantTok{FALSE}\NormalTok{) }\SpecialCharTok{+} 
    \FunctionTok{coord\_flip}\NormalTok{() }\SpecialCharTok{+} \FunctionTok{facet\_wrap}\NormalTok{(}\SpecialCharTok{\textasciitilde{}}\NormalTok{sentiment, }\AttributeTok{scales =} \StringTok{"free\_y"}\NormalTok{) }\SpecialCharTok{+} 
    \FunctionTok{ggtitle}\NormalTok{(}\StringTok{"Contribución al sentimiento"}\NormalTok{) }\SpecialCharTok{+} \FunctionTok{xlab}\NormalTok{(}\ConstantTok{NULL}\NormalTok{) }\SpecialCharTok{+} \FunctionTok{ylab}\NormalTok{(}\ConstantTok{NULL}\NormalTok{)}\SpecialCharTok{+}
  \FunctionTok{theme}\NormalTok{()}
\end{Highlighting}
\end{Shaded}

\includegraphics{Analisis_De_Sentimientos_Joaquín_Fernández_Suárez_files/figure-latex/unnamed-chunk-13-1.pdf}

Se puede comprobar como existe una mayor cantidad de palabras positivas
que negativas, lo que nos indica que los comentarios tienden a ser más
favorables al influencer y al contenido de la publicción.

Ahora clasificaremos las palabras según el sentimiento que describan,
sea ira, tristeza, alegria, sorpresa\ldots{} Para ello usaremos
\texttt{syuzhet} para poder trabajar en este tipo de análisis.

\begin{Shaded}
\begin{Highlighting}[]
\FunctionTok{library}\NormalTok{(syuzhet)}
\end{Highlighting}
\end{Shaded}

\begin{verbatim}
## Warning: package 'syuzhet' was built under R version 4.2.3
\end{verbatim}

\begin{Shaded}
\begin{Highlighting}[]
\NormalTok{sentimientos\_df }\OtherTok{\textless{}{-}} \FunctionTok{get\_nrc\_sentiment}\NormalTok{(token\_text}\SpecialCharTok{$}\NormalTok{word, }\AttributeTok{lang=}\StringTok{"spanish"}\NormalTok{)}
\end{Highlighting}
\end{Shaded}

\begin{verbatim}
## Warning: `spread_()` was deprecated in tidyr 1.2.0.
## i Please use `spread()` instead.
## i The deprecated feature was likely used in the syuzhet package.
##   Please report the issue to the authors.
## This warning is displayed once every 8 hours.
## Call `lifecycle::last_lifecycle_warnings()` to see where this warning was
## generated.
\end{verbatim}

Veamos cuales son los sentimientos que expresan los fans de elXokas en
sus comentarios.

\begin{Shaded}
\begin{Highlighting}[]
\FunctionTok{barplot}\NormalTok{(}
  \FunctionTok{colSums}\NormalTok{(sentimientos\_df[, }\DecValTok{1}\SpecialCharTok{:}\DecValTok{8}\NormalTok{]),}
  \AttributeTok{las =} \DecValTok{1}\NormalTok{,}
  \AttributeTok{cex.names =} \FloatTok{0.7}\NormalTok{,}
  \AttributeTok{main =} \StringTok{"Emociones de los comentarios de elXokas"}\NormalTok{,}
  \AttributeTok{xlab=}\StringTok{"emociones"}\NormalTok{, }\AttributeTok{ylab =} \ConstantTok{NULL}\NormalTok{)}
\end{Highlighting}
\end{Shaded}

\includegraphics{Analisis_De_Sentimientos_Joaquín_Fernández_Suárez_files/figure-latex/unnamed-chunk-15-1.pdf}
Se puede ver como el sentimiento de la confianza \textit{"trust"} esta
bastante presente en los comentarios análizados lo que puede indicar que
la comunidad del influencer se siente bastante cómoda con la
publicación, aunque es destacable la gran cantidad de palabras que se
relacionan con el miedo \textit{"fear"} teniendo en cuenta el tipo de
foto que se ha subido en sus redes.

\hypertarget{conclusiuxf3n}{%
\section{Conclusión}\label{conclusiuxf3n}}

Tras analizar los sentimientos y emciones de esta muestra de comentarios
de la publicación, se puede observar que existe bastantes palabras
positivas con respecto a la publicación, aunque extraña el tipo de
emociones que genera el influencer teniendo en cuenta el tipo de imagen
subida. Será interesante recopilar todos los comentarios para verificar
la información obtenida con esta muestra.

\end{document}
